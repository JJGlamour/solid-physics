\documentclass[UTF8]{article}
\usepackage{ctex}
\usepackage{setspace}
\usepackage{graphicx}
\usepackage{amsmath,xcolor}
\usepackage{cases}
\usepackage{array}
\usepackage{listings}
\usepackage{hyperref}
\hypersetup{
    colorlinks=true,
    linkcolor=blue,
    filecolor=blue,      
    urlcolor=blue,
    citecolor=cyan,
}
\lstset{breaklines}
\lstset{extendedchars=false}
\lstset{
    rulesepcolor= \color{gray}, %代码块边框颜色
    breaklines=true,  %代码过长则换行
    numbers=left, %行号在左侧显示
    numberstyle= \small,%行号字体
    keywordstyle= \color{blue},%关键字颜色
    commentstyle=\color{gray}, %注释颜色
    frame=shadowbox%用方框框住代码块
    }
\setcounter{MaxMatrixCols}{14}
\usepackage{geometry}
\setlength{\arraycolsep}{0.5pt}
\geometry{left=3.0cm,right=3.0cm,top=2.5cm,bottom=2.5cm}
\title{模拟一维单原子链和双原子链的振动}   %输入标题
\author{171870624 郭俊杰\\南京大学$\quad$现代工程与应用科学学院}   %输入名字
\date{\today}    %日期
\begin{document}
\maketitle

\begin{abstract}
在周期晶格中存在类波解,由于晶格的不连续性,波的振幅只在原子的格点上有定义,也就是格波。一个包括3N个自由度的周期性结构,存在3N个独立的简正模,等价于3N个独立的格波。在一维单原子链和一维双原子链中主要考虑最简单的晶格模型,即每个元胞中只有一个原子和两个原子。本文主要在计算求得一维单原子链和双原子链的色散关系的基础之上,通过其各自的格波解对原子偏移平衡位置情况进行模拟。程序采用JavaScript编写,可在web上直接查看振动效果,且可动态调节波矢q和双原子质量比m/M。 \\
\par\textbf{关键词: } 格波;一维单原子链;一维双原子链;色散关系;JavaScript
\end{abstract}


\section{引言}
原子结合成晶体,当原子距离增大时,会出现某种吸引力;当原子距离减小时,会出现由于泡利不相容原理引起的电子云的排斥力,这两者之间的平衡决定了原子的平衡间距,这是在T=0K时静态的研究,而在有限温度下,原子都是在平衡位置附近做振动,从此种原子振动的角度出发讨论晶体的宏观物理性质,就是晶格动力学。原子之间的振动是相互联系在一起的,这种联系在一起的振动在晶体中格点处形成各种模式的波,也就是格波。在振动非常弱的时候,这种相互作用可以认为是简谐的。

\subsection{一维单原子链}
一维单原子链为最简单的晶格模型,原子质量为M,间距为a,原子沿链的方向振动,第n个原子其偏离平衡位置的位移为$u_n$,$\beta_p$为距离为p的两原子之间的力常数,在简谐近似下,第n个原子受到的合力为$$F_n = \sum_p\beta_p(u_{n+p}-u_n)$$根据牛顿力学,第n个原子的运动方程是$$M\frac{\mathrm{d}^2u_n}{\mathrm{d}t^2}=\sum_p\beta_p(u_{n+p}-u_n)$$只考虑最近邻近似,即p=$\pm$1,且考虑到平移对称性有$\beta_{+1}=\beta_{-1}=\beta$,则原子运动方程简化为$$M\frac{\mathrm{d}^2u_n}{\mathrm{d}t^2}=\beta(u_{n+1}+u_{n-1}-2u_{n})$$其中n=0,$\pm$1,$\pm$2,...,$\pm\infty$,再考虑格波解$$u_l=Ae^{i(qna-\omega t)}$$将其带入最紧邻近似下的原子运动方程有$$-M\omega^2Ae^{i(qna-\omega t)}=\beta(e^{iqa}+e^{-iqa}-2)Ae^{i(qna-\omega t)}$$ $$\omega^2=2\frac{\beta}{M}(1-cosqa)=4\frac{\beta}{M}sin^2(\frac{qa}{2})$$于是得到色散关系$$\omega(q)=2\sqrt{\frac{\beta}{M}}\vert{sin(\frac{qa}{2})}\vert$$最后的频率与n无关,即无穷多个联立方程归结为同一解,也就是满足一定$\omega$与q关系的格波解。
\par
一个确定的q和$\omega(q)$确定系统的一个简正模,且相隔pa的原子振动有确定的相位差$$\frac{u_{n+p}}{u_n}=e^{iqpa}$$它表示该格波的位形。且q与q+$K_h$代表同一振动模,因此q取在第一布里渊区范围内,即$-\frac{\pi}{a} \textless q\leq\frac{\pi}{a}$。
\par
一维单原子链中频谱成能带结构,一个单原子链,相当于弹性波的低通滤波器。其本征频率必须限制在一个范围内,不存在超过最高频的本征频率。

\subsection{一维双原子链}
一维双原子链是另一种简单的晶格模型,由质量为M和m($M\textgreater m$)的原子组成双原子链。这是一种复式晶格,每个初基元胞中有两个不同的原子。令晶格常数为a,只考虑最紧邻作用,则两者的位移分别为$$M\frac{\mathrm{d}^2u_n}{\mathrm{d}t^2}=\beta(v_n+v_{n-1}-2u_n)$$ $${}m\frac{\mathrm{d}^2v_n}{\mathrm{d}t^2}=\beta(u_n+u_{n+1}-2v_n)$$
再根据格波解$$u_n=Ae^{i(qna-\omega t)}$$ $$v_n=Be^{i(qna-\omega t)}$$带入运动方程以后得到$$-\omega^2MA=\beta B(1+e^{-iqa})-2\beta A$$ $$-\omega^2mB=\beta A(1+e^{iqa})-2\beta B$$
要使得该方程有解,则系数行列式为0
$$\\
\left|\begin{array}{cccc} 
    2\beta-\omega^2    &    -\beta(1+e^{-iqa})     \\ 
    -\beta(1+e^{iqa})  &    2\beta-m\omega^2        \\ 
\end{array}\right| 
=Mm\omega ^4-2\beta(M+m)\omega^2+4\beta^2sin^2\frac{qa}{2}$$
于是得到频率$$\omega_\pm^2=\beta\frac{m+M}{mM}{\{1\pm[1-\frac{4mM}{(m+M)^2}sin^2\frac{qa}{2}]\}}$$
再将频率带入运动方程,即得到轻重原子的振幅之比$$\alpha_\pm={(\frac{B}{A}})_\pm=-\frac{M\omega_\pm^2-2\beta}{\beta(1+e^{-iqa})}$$
\par
其中$\omega_-$这一支称为声学支,$\omega_+$这一支称为光学支。对于小q,即在布里渊区中心处,声学支的波的群速度等于相速度:$\frac{\mathrm{d}\omega}{\mathrm{d}q}=\frac{\omega}{q}=c$,且其与频率无关,表现为长波长弹性波,两原子振幅之比$\alpha_-=\frac{B}{A}\approx1$;对于光学支,布里渊区中心处$\omega_+=\sqrt{\frac{2\beta}{\mu}}$,原子振幅之比$\alpha_+=\frac{B}{A}=\frac{M\omega_+^2-2\beta}{2\beta}\approx-\frac{M}{m}$,可见光学支在布里渊区中心处频率由力常数和折合质量决定,同时轻重原子振动方向相反,即两者相对运动而质心保持不动。而在布里渊区边界处,即q接近$\pm\frac{\pi}{a}$时,
$$\omega_-=\sqrt{\frac{2\beta}{M}},\quad\alpha_-=\frac{B}{A}\approx0$$
$$\omega_+=\sqrt{\frac{2\beta}{m}},\quad\alpha_+=\frac{B}{A}\approx\infty$$
显而易见,声学支只有重原子在振动,轻原子几乎不动,振动频率只与M有关;而光学支只有轻原子在振动,重原子保持不动,频率只与m有关。


\section{程序的设计}

\subsection{一维单原子链参数选取}
经上述推导计算已得出一维单原子链色散关系$$\omega(q)=2\sqrt{\frac{\beta}{M}}\vert{sin(\frac{qa}{2})}\vert$$在不考虑量纲的情况下,取晶格常数a=1,为其在页面上显示频率易于肉眼可见,取$2\sqrt{\frac{\beta}{M}}=\frac{1}{50}$,为换算到像素单位(px)再取A=70,于是根据$$\omega(q)=\frac{1}{50}\vert{sin(\frac{q}{2})}\vert,\quad u_n=70e^{i(qna-\omega t)}=70[cos(qna-\omega t)+isin(qna-\omega t)]$$实际模拟时从t=0时刻开始,每隔1ms根据$u_n$的实部计算一次每个原子偏移的位置,以像素为单位在屏幕上使原子移动相应的距离,随着时间的逐步增加,即可在页面上看到原子连续的振动效果。每次调节波矢q,则系统再次从t=0时刻开始新的模拟。控制原子移动的代码如下:
\begin{lstlisting}
        const q = Math.PI;
        let t = 0;
        let elements = [];
        function moveOneAtom(elem, y) {
            elem.style.bottom = y + "px";
        }
        for(let i=0; i < this.state.n; i++){
            elements.push(document.querySelector(`#atom${i}`));
        }
        this.timerID = setInterval(() => {
            for(let i = 0; i < this.state.n; i++) {
                let y = Math.cos(q *(i+1)  - (1/50) * Math.abs(Math.sin(q/2)) * t);
                console.log(`${i}原子:${String(y).substr(0,8)}`);
                moveOneAtom(elements[i], y * 70);
            }
            t++;
        }, 1);
\end{lstlisting}

\subsection{一维双原子链的参数选取及近似处理}
一维双原子链色散关系为$$\omega_\pm^2=\beta\frac{m+M}{mM}{\{1\pm[1-\frac{4mM}{(m+M)^2}sin^2\frac{qa}{2}]\}}$$同样在无量纲的情况下,取晶格常数a=1,重原子质量M,于是$\frac{m}{M}=m$,m取值范围为$0\textless m \textless1$,为显示直观,再取力常数$\beta=\frac{1}{1000}$,于是色散关系为
$$\omega_\pm^2=\frac{1}{1000}\frac{m+1}{m}{\{1\pm[1-\frac{4m}{(m+1)^2}sin^2\frac{q}{2}]\}}$$
显然其中包含两个可动态调节参数:质量比m/M=m和波矢q。
\par
在计算振幅时候采取近似处理,对于声学支:布里渊区中心处$\alpha_-=\frac{B}{A}\approx1$,布里渊区边界处$\alpha_-=\frac{B}{A}\approx0$,于是取A=70,则B=$70(-\vert\frac{q}{\pi}\vert+1)$,于是声学支M和m原子的格波解为
$$u_n=70e^{i(qn-\omega_- t)}=70[cos(qn-\omega_- t)+isin(qn-\omega_- t)]$$
$$v_n=70(-\frac{q}{\pi}+1)e^{i(qn-\omega_- t)}=70(-\frac{q}{\pi}+1)[cos(qn-\omega_- t)+isin(qn-\omega_- t)]$$
仍然从t=0时刻开始,根据$u_n$和$v_n$的实部每隔1ms计算一次各个原子的偏移量,然后以像素为单位使原子位置在页面上重新渲染,即可得到原子链直观的振动现象。
\par
对于光学支:布里渊区中心处$\alpha_+=\frac{B}{A}=\frac{M\omega_+^2-2\beta}{2\beta}\approx-\frac{M}{m}=-\frac{1}{m}$,布里渊区边界处$\alpha_+=\frac{B}{A}\approx\infty$,于是取B=70,A=$70(\frac{m}{\pi}\vert q\vert -m)$,根据格波解
$$u_n=70(\frac{m}{\pi}\vert q\vert -m)e^{i(qn-\omega_+ t)}=70(\frac{m}{\pi}{\vert q\vert} -m)[cos(qn-\omega_+ t)+isin(qn-\omega_+ t)]$$
$$v_n=70e^{i(qn-\omega_+ t)}=70[cos(qn-\omega_+ t)+isin(qn-\omega_+ t)]$$
仍然从t=0时刻开始,根据$u_n$和$v_n$的实部每隔1ms计算一次各个原子的偏移量,然后以像素为单位使原子位置在页面上重新移动渲染。至此即可得到可调节参数q$(-\pi,\pi)$和$\frac{m}{M}(0,1)$的双原子链的振动情况。另外程序还支持调节原子个数n(对数)以及其他页面显示效果(如原子间距、原子大小)。


\section{程序使用说明}
程序也网页形式展示,可在线查看模拟效果:\href{http://47.101.50.208}{http://47.101.50.208},另外程序还提供任意元胞中任意原子直角坐标和分数坐标转换的功能,本文中未做详细介绍。
\par
程序源代码地址:\href{https://github.com/JJGlamour/solid-physics}{https://github.com/JJGlamour/solid-physics},本程序兼容除IE以外的各大主流浏览器。



\begin{thebibliography}{99}  
\bibitem{ref1}胡安, 章维益. 固体物理学, 高等教育出版社, 第二版, 2011.  
\end{thebibliography}



\end{document}